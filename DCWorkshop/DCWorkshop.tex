%% v3.0 [2015/11/14]
\documentclass[paper]{ieicej}
\usepackage{graphicx}
\usepackage[T1]{fontenc}
\usepackage{lmodern}
\usepackage{textcomp}
\usepackage{latexsym}
\usepackage{amsmath}


\def\IEICEJcls{\texttt{ieicej.cls}}
\def\IEICEJver{3.0}
\newcommand{\AmSLaTeX}{%
 $\mathcal A$\lower.4ex\hbox{$\!\mathcal M\!$}$\mathcal S$-\LaTeX}
\newcommand{\PS}{{\scshape Post\-Script}}
\def\BibTeX{{\rmfamily B\kern-.05em{\scshape i\kern-.025em b}\kern-.08em
 T\kern-.1667em\lower.7ex\hbox{E}\kern-.125em X}}

\field{A}
\jtitle[SMTを用いた制約付きのロケーティングアレイの生成について]%
       {SMTを用いた制約付きのロケーティングアレイの生成について}
\etitle{On the generation of constrained locating arrays using an SMT solver}
\makeatletter
\if@paper
\makeatother
 \authorlist{%
  \authorentry[k-kou@ist.osaka-u.ac.jp]{金 浩}{Hao JIN}{OU,AIST}
   \MembershipNumber{58220953b9}
  \authorentry{崔 銀惠}{Eunhye CHOI}{AIST}\MembershipNumber{}
  \authorentry{土屋 達弘}{Tatsuhiro TSUCHIYA}{OU}\MembershipNumber{}
 }
\else
 \authorlist{%
  \authorentry[k-kou@ist.osaka-u.ac.jp]{金 浩}{Hao JIN}{OU,AIST}
   \MembershipNumber{58220953b9}
  \authorentry{崔 銀惠}{Eunhye CHOI}{AIST}\MembershipNumber{}
  \authorentry{土屋 達弘}{Tatsuhiro TSUCHIYA}{OU}\MembershipNumber{}
 }
\fi
\affiliate[OU]{大阪大学情報科学研究科,吹田市}
 {Graduate School of Information Science and Technology, Osaka University,
  Yamadaoka 1--5, Suita-shi, Osaka,
  565--0871 Japan}
\affiliate[AIST]{産業総合技術研究所,池田市}
 {Information Technology Research Institute, AIST Kansai,
  Midorigaoka 1--8--31, Ikeda-Shi, Osaka,
  563--8577 Japan}

\begin{document}
\makeatletter
\if@letter
\makeatother
\maketitle
\fi
\begin{abstract}
ソフトウェアに対するインタラクションテストにおいて,ロケーティングアレイをテスト集合として用いることで,故障検出だけでなくその特定も可能となる.本研究では,テストパラメータ上の制約を考慮したロケーティングアレイの生成を,SMTソルバを用いて行う方法について説明する.
\end{abstract}
\begin{keyword}
ロケーティングアレイ,組合せテスト,SMTソルバ
\end{keyword}
\begin{eabstract}
IEICE (The Institute of Electronics, Information and Communication Engineers)
provides a p\LaTeXe\ class file, named \IEICEJcls\ (ver.\,\IEICEJver),
for the IEICE Transactions. This document describes how to use
the class file, and also makes some remarks about typesetting
a manuscript by using the p\LaTeXe.
The design is based on ASCII Japanese p\LaTeXe.
\end{eabstract}
\begin{ekeyword}
p\LaTeXe\ class file, typesetting, math formulas
\end{ekeyword}
\makeatletter
\if@letter
\makeatother
\else
 \maketitle
\fi

\section{まえがき}
\label{sec:intro}
%組合せテストへの導入
ソフトウェアの開発において,その不具合,つまりフォールトを検出するためのデバッグに莫大なコストが費やされている.ソフトウェアシステムが日々複雑化している現在において,高効率な不具合検出手法はその重要性を著しく表し,具体的な検出手法の一つとして,組合せテスト(combinatorial testing)は注目を集めている.組合せテストはあらゆるk個のパラメータに対し,それらが取りうる値の組合せ(interaction)をすべて網羅することによって,フォールトに繋がるパラメータ値の組合せを検出できる.その根拠として,Kuhnらが2002年\cite{kuhn2002}及び2004年\cite{kuhn2004}に発表された論文により,大多数のソフトウェアの不具合はパラメータ数$k$が6以下の値の組み合わせにより生じることが挙げられる.組合せテストを利用することで,複数のテストケース(test case)が生成され,それらの集合をテストスイート(test suite)と名付ける.それぞれのテストケースは各パラメータに値を振り当てることであり,一つのベクトルとなる.そして,それらのベクトルの集合がテストスイートであり,マトリックスの形式で表す.

%組合せテストの基本な考え
組合せテストの核となる部分は「$k$個のパラメータの値の組み合わせを全て網羅する」であることは明らかであり,このような網羅に対する評価を$k$-wayカバレッジ($k$-way coverage)と呼ぶ.しかし,全てのパラメータ値の組合せを網羅したとしても,テスト集合のサイズが大き過ぎては,そのアドバンテージは現れない.いかに$k$-wayカバレッジの条件を満たしつつ,テストケース数を削減し,テストスイートのサイズを圧縮するかが焦点である.当問題は組合せ最適化問題の類に属し,先行研究では,様々な組合せテスト生成手法が提案されている.\cite{AETG}を始めとする貪欲法による生成方法もあり,他にも\cite{SATpairwise}のようなSATソルバを用いた生成法も存在する.

%ロケーティングアレイの簡単な説明,具体的には第二章で定義する
さらなる効果を求めるため,組合せテストによる検出機能のみならず,フォールトを特定する機能をも持つロケーティングアレイ(locating array)が提案されている\cite{colbourn}.組合せテストの特徴はパラメータ値の組合せを網羅することであり,その結果として生成されたテストスイートの中,たとえフォルス(false)になるテストケースを発見しても,そのテストケースが含む全ての組合せがフォールティインタラクション(faulty interaction)の候補であるため,そのさらなる特定は難しい.そこで,テストケースを増加することで,同一テストケースに含まれる組合せ間でも,他のテストケースと比較することにより,特定できるようなテストスイート,ロケーティングアレイが望ましい.フォールティインタラクション数が1であり,2-wayカバレッジを満たすロケーティングアレイを(1, 2)-locating arrayと呼ぶ.

%論文の構成
本論文の章構成は以下の通りである.{\bfseries \ref{sec:cla}} で制約付きロケーティングアレイについて述べ,{\bfseries \ref{sec:SMT}}ではSMT問題への変換について説明する.{\bfseries\ref{sec:experiments}} では提案手法に基づく実験を行い,実行時の動作結果について考察する.最後に{\bfseries \ref{sec:conclusion}} では,研究の結論と今後の展望について述べる.

\section{制約付きロケーティングアレイ}
\label{sec:cla}

{\bfseries\ref{sec:intro}}で述べたように,ロケーティングアレイは組合せテストのカバー特性を保ちつつ,テストケースを増やすことにより,さらなる特定機能を得たテストスイートである.これまでに,ロケーティングアレイについての研究は\cite{KonishiSAT}と\cite{Nagamoto}などが存在する.しかし,対象システム(SUT,System Under Testing)の固有制約を考慮したロケーティングアレイの研究はまだ少ない.一般的に,ほとんどのシステムはその仕様により,入力パラメータまたはコンポーネント間で固有の制限,制約(constraint)が存在する.これらを考慮することで,ロケーティングアレイはより実用化となる.本章ではそのような制約を考慮したロケーティングアレイ,(1, 2)-constrained locating array(CLA)の生成について述べる.

\subsection{基本概念}
\label{subsec:concept}
\begin{itemize}
\item SUTモデル

制約付きロケーティングアレイに適するシステムSUTを以下のように定義する.SUTはタプル$\langle\cal F,S,\phi\rangle$で表す,そのうち,${\cal F}=\{F_1, F_2, ..., F_k\}$と定義し,SUTにおける$k$個の入力パラメータを表す;各パラメータにおける変域の集合を${\cal S}=\{S_1, S_2, ..., S_k\}$と定義する;${\cal\phi}:S_1\times ... \times S_k\to \{true, false\}$はSUTモデルの固有制約を表す.各パラメータ変域には,二値以上の値を持ち,それらを0からの番号を振り当てることで,記述上の便利を図る,すなわち,$S_i=\{0, 1, ..., |S_i-1|\}$のように表す.以上のように定義することから,各テストケース$\boldsymbol\sigma$は$S_1\times S_2 \times ...\times S_k$の形をした一つのベクトルとなる.$N$個のテストケースを含むテストスイート$A$は,$N\times k$のアレイである.

\item 無効な値の組合せ

システム定義の$\cal \phi$により,特定なパラメータの値の組合せは有効ではないことがある.このような組合せは無効な値の組合せ(invalid interaction)と呼び,テスト設計では,テストケースに出現してはいけない.例を挙げると,システム$\langle\cal F,S,\phi\rangle$において,$|{\cal F}|=5, |S_1|=...=|S_5|=2$,パラメータ$F_1$及びパラメータ$F_2$は以下のような制約$\phi_1$が存在する:
\[ \phi_1 : F_1  =  0 \to F_2  =  0 \]
すなわち,$(F_1, F_2)$の組合せ$\{(0, 0), (0, 1), (1, 0), (1, 1)\}$のなか,$(1, 0)$は$\phi_1$と矛盾しているため,テストケースの中では出現できない.故に,このシステムにおいては,$(1, 0)$は無効な値の組合せの一つとなる.$\{(0, 0), (0, 1), (1, 1)\}$のようなインタラクションは有効な値の組合せ(valid interaction,以降VIとする)であり,すべての有効な二値の組合せ集合を${\cal VI}_2$と示す.そして,このようにパラメータに値を振り当てることを$\sigma(F_i, s_i)$と示す.テストケースにおいては,VIのみを含むものが有効である.テストスイート$A$のなか,有効なインタラクション$T$を含むテストケースの集合を$\rho_A(T)$と表す.

\item 区別不可能なインタラクションペア

テストケースを実行し,その結果からフォルティインタラクションを特定するには,以下の式を満たす必要がある:
\begin{eqnarray}
\label{CLA-core}
\forall T_1, T_2\in {\cal VI}_2; T_1 \not= T_2\Leftrightarrow\rho_A(T_1) \not= \rho_A(T_2)
\end{eqnarray}
故に,任意の異なる2つのインタラクションは,少なくとも一つのテストケースでは,同時に出現してはいけない.しかし,異なるVI間でも,固有制約により,上記の論理式を満たせない場合がある:
\[
T_1, T_2\in {\cal VI}_2\land T_1 \not= T_2\land\rho_A(T_1) = \rho_A(T_2)\nonumber
\]
このような2つのVIは(1, 2)-CLAにおいて特定不可能であるため,区別不能なインタラクションペア(indistinguishable pair,以降IPとする)と定義する.前例同様に,$|{\cal F}|=5, |S_1|=...=|S_5|=2$のようなシステムを考える.ここで,インタラクション$(\sigma(F_1, 0), \sigma(F_2, 0))$と$(\sigma(F_1, 0), \sigma(F_3, 0))$はVIとする.システムの制約$\phi$に以下のような2つの制約を想定する:
\begin{eqnarray}
\phi_2 : F_2 = 0 \to F_3 = 0
\end{eqnarray}
\begin{eqnarray}
\phi_3 : F_3 = 0 \to F_2 = 0
\end{eqnarray}
つまり,パラメータ$F_2, F_3$においては,片方が0の値を取る場合,もう片方も必ず0の値を取るということである.$\phi_2, \phi_3$により,$(\sigma(F_1, 0), \sigma(F_2, 0))$と$(\sigma(F_1, 0), \sigma(F_3, 0))$の2つのVIは必ず同一のテストケースに出現しなければならない.故に,この両者はIPである.すべてのVIに対し,2つのインタラクションを選択し,それらがIPではない場合は,少なくとも一つのテストケースに同時出現させないことで,(\ref{CLA-core})を満たすことができる.

\end{itemize}
\subsection{適応事例}
\label{subsec:example}

\begin{table}[tb]
	\caption{SUT:携帯電話}
	\label{sut:mobil}
	\begin{center}
		\begin{tabular}{|l|p{1cm}|p{1.35cm}|p{1.1cm}|p{1.15cm}|p{1.15cm}|}
			\hline
			$\cal F$ & $F_1$: Display & $F_2$: Email Viewer & $F_3$: Camera & $F_4$: Video Camera & $F_5$: Video Ringtones\\ \hline
			$\cal S$ & 0:16MC & 0:Graphical & 0:2MP & 0:Yes & 0:Yes \\ 
			& 1:8MC & 1:Text & 1:1MP & 1:No & 1:No \\
			& 2:BW & 2:None & 2:None &   &   \\ \hline
			$\cal \phi$ & \multicolumn{5}{|l|}{$\phi_1: F_2 = 0 \to F_3 \not= 2$}\\
			& \multicolumn{5}{|l|}{$\phi_2: F_3 = 0 \to F_1 \not= 2$}\\
			& \multicolumn{5}{|l|}{$\phi_3: F_2 = 0 \to F_3 \not= 0$}\\
			& \multicolumn{5}{|l|}{$\phi_4: F_1 = 1 \to F_3 \not= 0$}\\
			& \multicolumn{5}{|l|}{$\phi_5: F_4 = 0 \to (F_3 \not= 2 \land F_1 \not= 2)$}\\
			& \multicolumn{5}{|l|}{$\phi_6: F_5 = 0 \to F_4 = 0$}\\
			& \multicolumn{5}{|l|}{$\phi_7: \neg(F_1 = 0 \land F_2 = 1 \land F_3 = 0)$}\\
			\hline
		\end{tabular}
	\end{center}
\end{table}

\begin{table}[tb]
	\caption{無効なインタラクション}
	\label{InvalidInteractions}
	\begin{center}
		\begin{tabular}{|l|}
			\hline
			$((\sigma(F_1, 2), (\sigma(F_2, 0))$ \\
			$((\sigma(F_1, 1), (\sigma(F_3, 0))$ \\
			$((\sigma(F_1, 2), (\sigma(F_3, 0))$ \\
			$((\sigma(F_1, 2), (\sigma(F_4, 0))$ \\
			$((\sigma(F_1, 2), (\sigma(F_5, 0))$ \\
			$((\sigma(F_2, 0), (\sigma(F_3, 0))$ \\
			$((\sigma(F_2, 1), (\sigma(F_3, 0))$ \\
			$((\sigma(F_3, 2), (\sigma(F_4, 0))$ \\
			$((\sigma(F_3, 2), (\sigma(F_5, 0))$ \\
			$((\sigma(F_4, 1), (\sigma(F_5, 0))$ \\  
			\hline
		\end{tabular}
	\end{center}
\end{table}

\begin{table}[tb]
	\caption{区別不可能なインタラクションペア}
	\label{IndisPairs}
	\begin{center}
		\begin{tabular}{|l|}
			\hline
			$((\sigma(F_1, 0), (\sigma(F_3, 0))\ \sim\ ((\sigma(F_2, 2), (\sigma(F_3, 0))$ \\ 
			$((\sigma(F_1, 2), (\sigma(F_4, 1))\ \sim\ ((\sigma(F_1, 2), (\sigma(F_5, 1))$ \\
			$((\sigma(F_3, 2), (\sigma(F_4, 1))\ \sim\ ((\sigma(F_3, 2), (\sigma(F_5, 1))$ \\
			\hline
		\end{tabular}
	\end{center}
\end{table}

実用例として,\cite{eg:phone}に現れた携帯電話の例を利用する.SUTモデルは表\ref{sut:mobil}に示されている,$|{\cal F}| = 5, |{\cal S}| = 3^3 2^2, |\phi| = 7$.これらをもとに,{\bfseries \ref{subsec:concept}}で記述した順にインタラクションの有効性及びインタラクション間の区別可能性を確かめていく.

システムパラメータの値集合$\cal S$に対し,すべての2-wayインタラクションの集合を${\cal I}_2$とする.そして,制約$\cal \phi$から得られるすべてのVIを${\cal VI}_2$と示す.システム制約$\phi$による無効なインタラクションを表{\bfseries \ref{InvalidInteractions}}にて示す.${\cal VI}_2$に対して,IPは表{\bfseries \ref{IndisPairs}}のように得られる,これらを取り除き,残りのインタラクションペアを少なくとも一つのテストケースにおいて,同時出現しないように設計することで,(1, 2)-CLAを得ることができる.表{\bfseries \ref{sut:cla}}にてその実例を示す.

\begin{table}[tb]
	\caption{SUT: (1, 2)-CLA}
	\label{sut:cla}
	\begin{center}
		\begin{tabular}{|l|c c c c c|}
			\hline
			CLA & $F_1$ & $F_2$ & $F_3$ & $F_4$ & $F_5$ \\ \hline
			$\sigma_1$ & 0 & 0 & 1 & 0 & 0 \\
			$\sigma_2$ & 0 & 0 & 2 & 1 & 1 \\
			$\sigma_3$ & 0 & 1 & 1 & 0 & 1 \\
			$\sigma_4$ & 0 & 2 & 0 & 0 & 0 \\
			$\sigma_5$ & 0 & 2 & 0 & 0 & 1 \\
			$\sigma_6$ & 0 & 2 & 0 & 1 & 1 \\
			$\sigma_7$ & 0 & 2 & 2 & 1 & 1 \\
			$\sigma_8$ & 1 & 0 & 1 & 0 & 1 \\
			$\sigma_9$ & 1 & 0 & 1 & 1 & 1 \\
			$\sigma_{10}$ & 1 & 1 & 1 & 0 & 0 \\
			$\sigma_{11}$ & 1 & 1 & 2 & 1 & 1 \\
			$\sigma_{12}$ & 1 & 2 & 1 & 0 & 0 \\
			$\sigma_{13}$ & 2 & 1 & 1 & 1 & 1 \\
			$\sigma_{14}$ & 2 & 1 & 2 & 1 & 1 \\
			$\sigma_{15}$ & 2 & 2 & 1 & 1 & 1 \\
			\hline
		\end{tabular}
	\end{center}
\end{table}

\section{SMT問題への変換}
\label{sec:SMT}
SMT(satisfiability modulo theories)はSAT問題(satisfiability problems)に基づき,論理変数だけでなく,整数及びそれらの計算記号などをも扱える理論である.SMTソルバを利用することで,羅列した命題の充足可能な変数の値(解)を求められる.SAT/SMTソルバに関する研究はすでに捗っていて,高速に解を求めることができる.故に,SMTソルバのエンコーディングの容易さ,実行の素早さを利用することで,(1, 2)-CLAを求めることが効果的だと考えられる.本章では,{\bfseries \ref{subsec:example}}にて討論されたSUTモデルを対象にし,SMT-LIBの標準に基づき論議する.

\subsection{エンコーディング}
\label{subsec:encoding}
SMTソルバを用いて問題を解くために,まず変数を宣言しなければならない:
\[ {\rm (declare\mbox{\scriptsize-}const\ f1\ Int}) \]
上記のようにSUTの5つのパラメータを定義する.パラメータの値に関しては,断言(assertion)を加えることにより,その取りうる値を制限できる:
\[ {\rm (assert\ (and\ (>=\ f1\ 0)\ (<=\ f1\ 2))} \]
上記の断言から観察できるが,断言はすべてポーランド記法により記述される.変数宣言を実行した後,各パラメータの値の範囲,そして制約を順次に書き出す.

入力されたすべての断言に対し,真となる解が存在するのならば,SMTソルバは「SAT」の結果を出し,各変数の値を提示する.しかし,真となる解が存在しないのならば,つまり,断言間にて矛盾が存在するのならば,SMTソルバは「UNSAT」の結果を提示する.

\subsubsection{無効なインタラクション}
\label{subsec:invalid}
定義に基づき,無効なインタラクションはSUTモデルの固有制約$\phi$と矛盾しているため出現する.現にSMTソルバに変数の宣言,値の範囲,固有制約を入力したとする.すべての2-wayインタラクション${\cal I}_2$において,入力済みの断言と合わせ「UNSAT」の結果が出現した場合,そのインタラクションは無効だと判断できる:
\[{\rm (assert\ (and\ (=\ f1\ 2)\ (=\ f2\ 0)))}\]

すべてのインタラクションに対し,上記の断言をそれぞれ実行した結果,表{\bfseries \ref{InvalidInteractions}}の10個の無効なインタラクションを特定できる.これら無効なインタラクションの集合を${\cal I}_2$から取り除き,${\cal VI}_2$を求められる.

\subsubsection{区別不能なインタラクションペア}
\label{subsec:indistinguishable}
${\cal VI}_2$を得た後,集合中の任意な異なる2つのインタラクション$T_1, T_2$に対し,その区別可能性を確かめる.無効なインタラクションと同様,先にSMTソルバに変数宣言,値範囲及び固有制約を入力する.定義によると,区別可能性を確かめる方法は,同じテストケースに同時に出現させないことで,それが可能であれば,両者は区別可能である.この特性から,以下のような命題を書き出せる:
\[((F_1 = s_0) \land (F_2 = s_0)) \oplus ((F_1 = s_0) \land (F_3 = s_0))\]
上記の命題をSMT-LIB標準に基づき,書き換える:
\begin{eqnarray}
{\rm (assert\ (xor}&{\rm (and\ (=\ f1\ 0)\ (=\ f2\ 0))\ } \nonumber \\
 &{\rm \ (and\ (=\ f1\ 0)\ (=\ f3\ 0))))} \nonumber
\end{eqnarray}
この式をすべての${\cal VI}_2$ペアに応用する.「UNSAT」の結果を出すペアはIPである.これらを取り除き,残りのペアは区別可能のペアである,${\cal DP}$と記す.
\subsubsection{テストスイート構成}
\label{subsec:suiteconstruct}

\subsection{Symmetry Breaking}
\label{subsec:symmetry}

\section{実験・評価}
\label{sec:experiments}

\section{結論}
\label{sec:conclusion}

\begin{itemize}
\item
最終原稿の提出に関しては,各ソサイエティの
「投稿のしおり」を参照してください.
\item
ソース・ファイルはできるだけ1本のファイルにまとめてください.
\item
著者独自のマクロを記述したファイルや文献,
図の EPS ファイルなどを忘れていないかご確認願います.
\end{itemize}

\begin{thebibliography}{99}
%\bibitem{ohno}
%大野義夫編,\TeX\ 入門,
%共立出版,東京,1989.

%\bibitem{Seroul}
%R. Seroul and S. Levy, A Beginner's Book of \TeX,
%Springer-Verlag, New York, 1989.

%\bibitem{nodera1}
%野寺隆志,楽々\LaTeX{},
%共立出版,東京,1990.

%\bibitem{itou}
%伊藤和人,\LaTeX\ トータルガイド,
%秀和システムトレーディング,1991.

%\bibitem{nodera2}
%野寺隆志,今度こそ\AmSLaTeX{},
%共立出版,東京,1991.

\bibitem{kuhn2002}
D.E. クヌース,改訂新版 \TeX\ ブック,
アスキー出版局,東京,1992.

\bibitem{kuhn2004}
磯崎秀樹,\LaTeX\ 自由自在,
サイエンス社,東京,1992.

%\bibitem{impress}
%鷺谷好輝,日本語 \LaTeX\ 定番スタイル集,
%インプレス,東京,1992--1994.

%\bibitem{Gr}
%G. Gr\"{a}tzer,
%Math into \TeX\,--\,A Simple Introduction to \AmSLaTeX,
%Birkh\"{a}user, 1993.

\bibitem{AETG}
藤田眞作,
化学者・生化学者のための\LaTeX---パソコンによる論文作成の手引,
東京化学同人,東京,1993.

%\bibitem{styleuse}
%古川徹生,岩熊哲夫,
%\LaTeX\ のマクロやスタイルファイルの利用(styleuse.tex),1994.

\bibitem{SATpairwise}
阿瀬はる美,てくてく\TeX{},
アスキー出版局,東京,1994.

\bibitem{colbourn}
S. von Bechtolsheim, \TeX\ in Practice,
Springer-Verlag, New York, 1993.

\bibitem{KonishiSAT}
N. Walsh, Making \TeX\ Work,
O'Reilly \& Associates, Sebastopol, 1994.

\bibitem{Nagamoto}
D. Salomon, The Advanced \TeX\ book,
Springer-Verlag, New York, 1995.

\bibitem{eg:phone}
phone example

\end{thebibliography}

\appendix
%\label{sec:app}
\section{PDFの作成方法とA4用紙への出力}

\begin{itemize}
\item
PDF に書き出すには二通りの方法があります.
\begin{enumerate}
\item
dvipdfmx を使って PDF に変換する(以下では段幅の関係で折り返します).
\begin{verbatim}
dvipdfmx -p 182mm,257mm -x 1in -y 1in
 -o file.pdf file.dvi
\end{verbatim}
用紙サイズとして \texttt{jisb5} というオプションが使えるかもしれません.
オプションの \texttt{-x 1in -y 1in} は省略できます.

\item
まず,dvips を使用して,ps に書き出します.
\begin{verbatim}
dvips -Pprinter -t b5 -O 0in,0in
  -o file.ps file.dvi
\end{verbatim}
\texttt{printer} には,使用するプリンタ名を記述します.
オプションの \texttt{-O 0in,0in} は省略できます.

次に Acrobat Distiller で PDF に変換します.
\end{enumerate}

\item
\texttt{dvips} を使用してA4用紙に出力する場合の
パラメータはおおよそ以下のような設定になります.
\begin{verbatim}
dvips -Pprinter -t a4 -O 14mm,20mm file.dvi
\end{verbatim}
\texttt{printer} には使用するプリンタ名を記述します.
オプションの \texttt{-t a4} は省略できます.

\item
Windows 上の dviout を利用して,
用紙の左右天地中央に版面を自動配置する場合は以下のようにします.
\begin{enumerate}
\item
dviout を起動します.
\item
メニューバーにある Option の中の Setup Parameters... を選択します.
\item
「DVIOUTのプロパティ」というダイアログが開くので,
Paper というタブを選択します.
\item
Options という枠の中に Horizontal centering と Vertical centering という
チェックボックスがあるので,
それぞれチェックした後に Save ボタンを押します.
\item
この設定を行った後に dvi ファイルを開くと,
版面が用紙の中心に配置されます.
\end{enumerate}
\end{itemize}

%\section{\texttt{jis.tfm} の利用}
%\label{sec:jistfm}
%
%株式会社リーブルテック(旧東京書籍印刷)の小林肇さんが作成された
%\texttt{jis.tfm} の利用を奨めます.
%ドキュメントクラスのオプションに \texttt{usejistfm} を指定します.
%\begin{verbatim}
%\documentclass[paper,usejistfm]{ieicej}
%\end{verbatim}
%テンプレート(\texttt{template.tex})ではデフォルトの設定になっています.
%
%\texttt{jis.tfm} のインストールなどに関しては
%「日本語\TeX\ 情報」
%(http://oku.edu.mie-u.ac.jp\slash\~{}okumura\slash texfaq\slash{})
%などを参照してください.

\section{削除したコマンド}

本誌の体裁に必要のないコマンドは削除しています.
削除したコマンドは,\verb/\part/,\allowbreak
\verb/\theindex/,\allowbreak
\verb/\tableofcontents/,\allowbreak
\verb/\titlepage/,\allowbreak
ページスタイルを変更するオプション(\texttt{headings},
\texttt{myheadings})などです.

%\section{変更履歴}
%
%\subsection{v1.1(v1.0 からの変更点)}
%
%\begin{itemize}
%\item
%「技術研究報告」の体裁に対応.
%\item
%複数の所属がある場合に,ラベルの記述順に所属マークが出るように変更.
%\item
%C分冊の場合に,メールアドレスを脚注部分に出力できるようにした.
%\end{itemize}
%
%\subsection{v1.2(v1.1 からの変更点)}
%
%\begin{itemize}
%\item
%「技術研究報告」の先頭ページの出力様式を変更.
%\end{itemize}
%
%\subsection{v1.3(v1.2 からの変更点)}
%
%\begin{itemize}
%\item
%B--I,B--II および C--I,C--II 分冊がそれぞれ,
%B,C に統合されたことに伴う修正.
%\item
%C分冊で「レター論文」が「レター」と変更されたことに伴う修正.
%\item
%すべての分冊でメールアドレスを脚注部分に出力できるように修正.
%\item
%英文アブストラクトと英文キーワードを記述できるように定義を追加.
%\item
%複数の「現在の所属ラベル」を指定できるように修正.
%\item
%会員種別の「Associate Member」(准員)を「Affiliate Member」に,
%「Honorary Member」(名誉員)を「Fellow, Honorary Member」に変更.
%および,「正員:フェロー(Fellow)」を追加.
%\item
%\verb/\typeofletter/ を指定しない場合,
%「研究速報」の体裁になるように変更.
%\item
%著者紹介で顔写真の EPS を取り込めるようにした.
%顔写真のない場合にも対応.
%%\item
%% \verb/\finalreceived/(最終受付)を追加.
%\item
%\texttt{jis.tfm} が使用できるようにした.
%\end{itemize}
%
%\subsection{v1.4(v1.3 からの変更点)}
%
%このバージョンは配布されなかったようです.
%\begin{itemize}
%\item
%先頭ページのフッタに,
%「(社)電子情報通信学会」の copyright 表示を追加した.
%\end{itemize}
%
%\subsection{v1.5(v1.4 からの変更点)}
%
%\begin{itemize}
%\item
%\verb/\documentclass/ のオプション \texttt{letterpaper} を
%\texttt{electronicsletter} に変更した.
%\item
%D--I,D--II 分冊が統合されたことに伴う修正.
%\end{itemize}
%
%\subsection{v1.6(v1.5 からの変更点)}
%
%\begin{itemize}
%\item
%会員種別に「正員:シニア会員(Senior)」を追加.
%\end{itemize}
%
%\subsection{v2.0(v1.6 からの変更点)}
%
%\begin{itemize}
%\item
%会員番号を投稿原稿の最終ページに出力するため,
%\verb/\MembershipNumber/ を定義.
%
%\item
%\verb/\affiliate/ の記述を,所属/部署/住所に分けて記述するようにした.
%\end{itemize}
%
%\subsection{v3.0(v2.0 からの変更点)}
%
%\texttt{uplatex}によるコンパイルに対応した.

\makeatletter
\if@letter\else
\makeatother
\begin{biography}
\profile{m}{電子 花子}{%
1996東北一大学情報工学科卒.
1999東京第一大学工学部工学部助手.
某システムの研究に従事.
}
\profile{m}{情報 太郎}{%
1995大阪一大学工学科卒.
1997同大大学院工学研究科修士課程了.
1998大阪(株)入社.
某コンピュータ応用の研究に従事.
ABC学会会員.
}
\profile{n}{通信 次郎}{%
1980九州一大学理工学部卒.
1981大阪(株)入社.
現在ATT日本研究所に所属.
}
\end{biography}
\fi

\end{document}
